\documentclass{report}
\usepackage{geometry, graphicx, kotex, imakeidx, titlesec} % necessary packages
\usepackage{amsmath, amsthm, amssymb, mathrsfs, tabularx, multirow, verbatim, url,} % supplementary packages
\usepackage[table]{xcolor}

\definecolor{lgray}{gray}{0.95} % lightgray color
\titleformat{\chapter}{\normalfont\huge\bfseries}{\chaptertitlename\ \thechapter.}{20pt}{\huge} % chapter style
\newtheorem{theorem}{Theorem} % theorem environment
\newtheorem{definition}{Definition} % definition environment
% if you need similar environments like lemma, corollary or remark, add them all.
\makeindex % a command for making index chapter
\linespread{1.25} % set the vertical spacing between successive lines (the ratio of the maximal height of a normal font and the baselineskip)

%%% stand for a new chapter or a new page
%% stand for a new section
% stand for a new subsection or a comment

%%%%
\begin{document}
\pagenumbering{gobble} % disable the page number until otherwise specified.
\newgeometry{paper=b5paper, left=30mm, right=30mm, top=30mm, bottom=30mm} % set the paper size and the margins

%%% the cover page
\newpage
\noindent
\begin{tabularx}{\textwidth}{| >{\centering\arraybackslash}X |}
\arrayrulecolor{lgray}
\hline
\Large 석(박) 사 학 위 논 문 \\\hline
~\vspace{70pt}\\\hline % 3cm spacing % 0.5cm is approximately 14pt, where default vertical spacing is 0.5cm(14pt). %5\times14pt=70pt
\huge 학위 논문 제목\\ % put the title here
\Large 부제가 있을 경우 중앙에 위치\\
~\vspace{14pt}\\\hline % 여백 조정 가능
\LARGE 고 려 대 학 교 ~ 대 학 원\\\hline % ~ for spacing
~\\\hline
\Large OOO 학과 \\\hline % put the department here
~\\\hline
\Large 홍 길 동 \\\hline % put the student name here
~\\\hline %3cm spacing %5\times14pt=70pt
\large 2023년 2월 \\\hline % put the graduation month here
\end{tabularx}

%%% an empty page
\newpage
~
%논문 표지 뒤에는 빈 페이지를 삽입합니다.
%
%본 양식은 대학원 학칙 일반대학원 시행세칙 및 일반대학원 학위논문 작성법에 따라서 MS Word를  사용하여 2022년 10월 작성되었다. 
%제공된  양식에서 논문의 순서 및 겉표지, 속표지, 심사완료검인서, 초록 페이지에 안내된 내용은 반드시 따라야 하는 부분이며, 나머지 부분들은 한가지 예시로 제공되었으므로, 수정하여 사용 가능하다. 예로 장, 절, 항 제목의 서체, 글자크기, 정렬방식은 적절하게 변경하여 사용할 수 있다. 나머지 부분에서 반드시 지켜야 할 내용, 예를 들어 본문 글자 사이즈(10-12 포인트), 페이지 매기기 등은 “일반대학원_학위논문_작성법” 파일을 참고하여야 한다.
%또한, 제공된 양식에서 설명 글에 해당하는 부분은 삭제하여야 하며, 겉표지, 속표지 등에 있는 테이블 줄은 보이지 않게 처리하여야 한다.

%%% the cover page
\newpage
\noindent
\begin{tabularx}{\textwidth}{| >{\centering\arraybackslash}X |}
\arrayrulecolor{lgray}
\hline
\Large 김 철 수 ~ 교 수 지 도 \\\hline % put the professor name here
~\\\hline % 0.5cm 여백
\Large 석(박) 사 학 위 논 문 \\\hline
~\vspace{35pt}\\\hline %2-3cm 여백 %3\times14pt=42pt
\huge 학위 논문 제목\\ % put the title here
\Large 부제가 있을 경우 중앙에 위치\\
~\vspace{28pt}\\\hline % 여백 조정 가능
\Large 이 논문을 O학 석(박)사학위 논문으로 제출함\\\hline % put the major here
~\vspace{35pt}\\\hline %2-3cm 여백 %3\times14pt=42pt
\large 2023년 2월\\\hline % put the graduation month here
~\vspace{35pt}\\\hline %2-3cm 여백 %3\times14pt=42pt
\LARGE 고 려 대 학 교 ~ 대 학 원\\\hline % ~ for spacing
~\\\hline % 0.5cm 여백
\Large OOO 학과 \\\hline
~\vspace{14pt}\\\hline %1cm 여백 1\times14pt=14pt
\Large 홍길동 (인) \\\hline % put the student name here
\end{tabularx}


%%% the approval page for master's thesis
\newpage
\noindent
\begin{tabularx}{\textwidth}{| >{\centering\arraybackslash}X |}
\arrayrulecolor{lgray}
\hline
~\vspace{14pt}\\\hline %1cm 여백 %1\times14pt=14pt
\Large 홍길동의 O학 석(박)사학위 논문 심사를 완료함\\\hline
~\vspace{42pt}\\\hline %2-3cm 여백 %3\times14pt=42pt
\Large 20OO년 O월\\\hline
~\vspace{42pt}\\\hline %2cm 여백 %3\times14pt=42pt
\Large 위\phantom{원}원\qquad\phantom{박사의 경우 추가 }\qquad (인)\\\hline
~\vspace{14pt}\\\hline %0.5cm 여백
\Large 위\phantom{원}원\qquad\phantom{박사의 경우 추가 }\qquad (인)\\\hline
~\vspace{14pt}\\\hline %0.5cm 여백
\Large 위\phantom{원}원\qquad\phantom{박사의 경우 추가 }\qquad (인)\\\hline
~\vspace{14pt}\\\hline %0.5cm 여백
\Large 위\phantom{원}원\qquad박사의 경우 추가 \qquad (인)\\\hline
~\vspace{14pt}\\\hline %0.5cm 여백
\Large 위\phantom{원}원\qquad박사의 경우 추가 \qquad (인)\\\hline
~\vspace{14pt}\\\hline %0.5cm 여백
\end{tabularx}

%%% the approval page for docgtoral thesis
\newpage 
\begin{center}
\Large
The dissertation of Gildong Hong has been approved \par % put the student name and the major here
by the dissertation committee in partial fulfillment\par
of the requirements for the degree of \par
Master of Philosophy (in Major) 
\par\vspace{50pt}
\large December 2022 % put the approval month here
\par\vspace{50pt}
{\small\color{gray}{signiture}}\\[-10pt]\rule{.6\textwidth}{0.4pt}\par
Committee Chair: Chulsu Kim % put the committee chair name here
\par\vspace{20pt}
{\small\color{gray}{signiture}}\\[-10pt]\rule{.6\textwidth}{0.4pt}\par
Committee Member: Name % put the committee member name here
\par\vspace{20pt}
{\small\color{gray}{signiture}}\\[-10pt]\rule{.6\textwidth}{0.4pt}\par
Committee Member: Name % put the committee member name here
\par\vspace{20pt}
{\small\color{gray}{signiture}}\\[-10pt]\rule{.6\textwidth}{0.4pt}\par
Committee Member: Name % put the committee member name here 
\par\vspace{20pt}
{\small\color{gray}{signiture}}\\[-10pt]\rule{.6\textwidth}{0.4pt}\par
Committee Member: Name % put the committee member name here 
\par\vspace{20pt}
{\small\color{gray}{signiture}}\\[-10pt]\rule{.6\textwidth}{0.4pt}\par
Committee Member: Name % put the committee member name here 
\end{center}

%%% an empty page
\newpage ~ 

%%% korean abstract page
\newpage 
\begin{center}
\LARGE 국문 제목 % put the title here
\par\vspace{20pt}
\normalsize by 홍길동\par % put the student name here
OO 학과\par % put the professor name and the department here
지도교수 : 김철수
\par\vspace{20pt}
\large \textbf{국문 초록}
\end{center}
\normalsize
국문 학위논문의 초록은 국문, 영문의 순서로 작성하며, 영문 학위논문의 초록은 영문, 국문의 순서로 작성하며, 학위논문을 기타 외국어로 작성하는 경우 초록은 기타 외국어, 영문, 국문의 순서로 작성한다. (표준 스타일 적용)

초록에는 논문제목, 성명, 학과, 지도교수를 기재하며 초록 하단에 주요어(keywords)를 표기한다. 
페이지 번호는 초록부터 본문 전까지 작은 로마 숫자(Roman numerals, e.g., i, ii, iii, iv...)를 사용한다.

\par\vspace{100pt}
\textbf{중심어} : 중심어, 중심어, 중심어, 중심어, 중심어, 중심어

%%%english abstract page
\newpage 
\pagenumbering{roman} % set the page number as roman type from this page on.
\newgeometry{paper=b5paper, left=20mm, right=20mm, top=30mm, bottom=30mm} % set the paper size and the margins
% 이 페이지부터 여백(아래쪽, 위쪽, 3cm, 오른쪽, 왼쪽 2cm) 변경됨
% Margins shall be changed to bottom and top 3 cm, right and left 2 cm from this page forward.
\addcontentsline{toc}{chapter}{초록}
\begin{center}
\LARGE Title % put the title here
\par\vspace{20pt}
\normalsize by Gildong Hong\par % put the student name here
Department of OOOO\par
under the supervision of Professor Chulsu Kim % put the professor name and the department here
\par\vspace{20pt}
\large \textbf{ABSTRACT}
\end{center}
\normalsize
The text of the abstract begins here. 
%The above title line (ABSTRACT) is styled using \large and  \textbf.
% Paragraph text is styled using default style.
% Pages should be assigned from the abstract using small Roman numericals (i, ii, iii, iv, v, etc.)
\par\vspace{100pt}
\textbf{Keywords} : Keyword, Keyword, Keyword, Keyword, Keyword, Keyword

%%% an empty page
\newpage 
~ 

%%% the dedication page
\newpage 
~
%감사의 글은 필요한 경우 작성한다. 감사의 글 작성시, 페이지 중앙에 위치하도록 한다. 감사의 글(Dedication) 제목은 생략하여 목차에 표현하지 않는 것이 일반적이다.

%%% the preface page
\chapter*{서문}
\addcontentsline{toc}{chapter}{서문}
The text of the preface begins here. 
%학위논문에 다른 사람들과 협력하여 수행된 결과가 포함되거나, 저자가 출판한 내용이 포함되는 경우, 이와 관련된 내용을 서문에 작성하여야 한다. 서문에는 아래의 내용이 포함될 수 있다. 다만, 서문을 따로 작성하지 않고, 관련 사항을 본문의 서론에서 언급하는 것도 가능하다. 
%① 다른 사람들과 협력하여 수행한 작업에 대한 다른 사람의 기여도와 비율 및 저자가 독창적이라고 주장하는 부분에 대한 설명
%② 논문의 일부분이 이미 출판되었거나 준비 중인 부분에 대한 설명 및 출판물에 대한 모든 사람의 기여
%③ 이외에도 논문작성 관련 개인적 상황 및 정보(personal information), 주제 선택 동기(motivation), 저자 관점, 감사 및 사사(acknowledgments) 등의 내용이 포함될 수 있다.
%
%\bigskip
%서문 작성 예
%\begin{itemize}
%\item\url{https://www.grad.ubc.ca/sites/default/files/doc/page/thesis_sample_prefaces.pdf}
%\item\url{https://www.phase-trans.msm.cam.ac.uk/2002/thomas/chapter1.pdf}
%\end{itemize}

%%% the acknowledgement page
\chapter*{사사}
\addcontentsline{toc}{chapter}{사사}
The text of the acknowledgements begins here.
%필요한 경우 사사를 작성한다. 
%서문(Preface)에서 사사(acknowledgments)와 관련된 내용을 기술한 경우, 생략할 수 있다.

%%% the table of contents page
\renewcommand{\contentsname}{목차}
\tableofcontents
%위의 목차는 스타일 “제목1, 2, 3”이 적용된 장, 절, 항 제목에 대해 자동적으로 생성된다.
%목차는 초록부터 작성한다.
%페이지 번호는 초록부터 본문 전까지 작은 로마 숫자(Roman numerals, e.g., i, ii, iii, iv...)를 사용한다. 본문의 서론부터 아라비아 숫자(Arabic numbers, e.g., 1, 2, 3...)를 사용한다.
%본문의 장은 아라비아 숫자(1, 2, 3, 4...), 부록은 영문 알파벳(A, B, C...)을 사용하여 구분하는 것이 일반적이다.

%%% the list of tables page
\renewcommand{\listtablename}{표 목차}
\listoftables
%위의 표 목차는 스타일 “Table Title”이 적용된 표 제목에 대해 자동적으로 생성된다.
%본문에 표가 포함되는 경우 반드시 표 목차를 작성한다. 본문을 한글로 작성하더라도 표 제목는 영어로 작성 가능하다. 표는 본문 전체에 대해 연속적인 번호를 부여(1, 2, 3, 4, 5...) 하거나, 각 장(Chapter)에 기반하여 번호를 부여(1.2, 1.2, 2.1, 2.2...) 할 수 있다.

%%% the list of figures page
\renewcommand{\listfigurename}{그림 목차}
\listoffigures
%위의 그림 목차는 스타일 “Table Title”이 적용된 그림 제목에 대해 자동적으로 생성된다.
%본문에 그림이 포함되는 경우 반드시 그림 목차를 작성한다. 본문을 한글로 작성한 경우에도 그림 제목는 영어로 작성 가능하다. 그림은 본문 전체에 대해 연속적인 번호를 부여(1, 2, 3, 4, 5...) 하거나, 각 장(Chapter)에 기반하여 번호를 부여(1.2, 1.2, 2.1, 2.2...) 할 수 있다.

%%% the nomenclature page
\chapter*{기호 설명}
\addcontentsline{toc}{chapter}{기호 설명}
\begin{tabular}{p{.2\textwidth}p{.7\textwidth}}
$M$	& original mass matrix\\
$K$	& original stiffness matrix\\[30pt]
\multicolumn{2}{l}{Subscripts}\\
$b$ & interface boundary\\
$d$ & dominant\\[30pt]
\multicolumn{2}{l}{Abbreviation}\\
$CMS$ & Component Mode Synthesis\\
\end{tabular}
%필요한 경우 기호설명을 작성한다. 기호 설명에는 필요한 경우 첨자 설명 및 약어 설명을 포함한다.
% 본문을 한글로 작성한 경우에도 기호 설명은 영어로 작성 가능하다.

%%% an empty page
\newpage 
~% 본문 앞에는 빈페이지를 둔다. 

%%% the first chapter of the main body
\pagenumbering{arabic} % set the page number as arabic type from this page on.
\chapter{서론}\label{chap:intro}
장(chapter)을 만들기 위해 \verb|\chapter{서론}|을 사용하였으며, 목차생성시 자동적으로 포함된다.
본문부터 페이지 번호는 아라비아 숫자(e.g., 1, 2, 3...)를 사용한다.
모든 장과 절(section), 항(subsection)의 첫 문단은 기본적으로 들여쓰기가 되지 않도록 되어 있다.

이 장을 라벨링 하기 위해서는 \verb|\label{chap:intro}|와 같은 명령어를 사용할 수 있다.

%%
\section{절 제목}\label{sec:section}
절(section)을 만들기 위해 \verb|\section{절 제목}|을 사용하였으며, 목차생성시 자동적으로 포함된다.

이 절을 라벨링 하기 위해서는 \verb|\label{sec:section}|와 같은 명령어를 사용할 수 있다.

%
\subsection{항 제목}\label{subs:subsection}
항(subsection)을 만들기 위해 \verb|\subsection{항 제목}|을 사용하였으며, 목차생성시 자동적으로 포함된다.

이 항을 라벨링 하기 위해서는 \verb|\label{subs:subsection}|와 같은 명령어를 사용할 수 있다.

장, 절, 항에 대한 더 많은 정보를 위해서는 다음과 같은 자료를 참고할 수 있다 ; \url{https://www.overleaf.com/learn/latex/Headers_and_footers}

%
\section{장, 절, 항을 참조하기}\label{sec:referencing}
위에 적은 첫번째 절을 참고하기 위해서는 \verb|\ref{sec:section}|와 같은 명령어를 사용할 수 있다 : 절 \ref{sec:section}

위에 적은 첫번째 항을 참고하기 위해서는 \verb|\ref{subs:subsection}|와 같은 명령어를 사용할 수 있다 : 항 \ref{subs:subsection}

상호참조에 대한 더 많은 정보를 위해서는 다음과 같은 자료들을 참고할 수 있다 :
\begin{itemize}
\item
\url{https://en.wikibooks.org/wiki/LaTeX/Labels_and_Cross-referencing}
\item
\url{https://www.overleaf.com/learn/latex/Cross_referencing_sections%2C_equations_and_floats}
\end{itemize}

%%%  the second chapter of the main body
\chapter{학위논문의 양식}\label{chap:organizing}

%%
\section{학위논문의 순서}\label{sec:order}
학위논문은 논문 표지, 속표지, 심사완료검인서, 초록, 감사의 글(선택), 서문(필요시), 사사(선택), 목차, 표목차(본문에 표가 포함된 경우), 그림목차(본문에 그림이 포함된 경우), 기호설명(선택), 본문, 참고문헌, 부록(선택), 색인(선택)의 순서로 한다.

%%
\section{용지 크기, 여백 및 페이지 설정} \label{sec:papersize}
논문의 규격은 4·6배판(B5)로 하는 것을 원칙으로 한다.

논문 표지, 속표지, 심사완료검인서의 아래쪽, 위쪽, 오른쪽, 왼쪽의 여백은 3cm 이상으로 한다. 초록부터 페이지 여백은 아래쪽, 위쪽, 3cm 이상 오른쪽, 왼쪽 2cm 이상으로 한다.

페이지 번호는 초록부터 본문 전까지 작은 로마 숫자(Roman numerals, e.g., i, ii, iii, iv...)를 사용하며, 본문의 서론부터 아라비아 숫자(Arabic numbers, e.g., 1, 2 , 3...)를 사용한다.

 표는 본문 전체에 대해 연속적인 번호를 부여(1, 2, 3, 4, 5...) 하거나, 각 장(Chapter)에 기반하여 번호를 부여(1.2, 1.2, 2.1, 2.2...) 할 수 있다. (Table \ref{tab:Organizing and formatting}).

\renewcommand\tablename{표}
\begin{table}[h]\centering
\begin{tabular}{cccc}
\hline
\textbf{순서}&\textbf{비고}&\textbf{여백설정}&\textbf{페이지설정}\\\hline
논문표지&&\multirow{4}{2.5cm}{\centering 위, 아래, 왼쪽 오른쪽 모두 3 cm 이상}&\multirow{4}{2.5cm}{\centering 없음}\\\cline{1-2}
빈페이지&&\\\cline{1-2}
속표지&&\\\cline{1-2}
심사완료검인서&&\\\hline
초록&국문 및 영문 초록 모두 작성 필요&\multirow{13}{2.5cm}{\centering  위, 아래 3 cm 이상\\[\baselineskip] 왼쪽, 오른쪽 2cm 이상}\\\hline
감사의 글&선택&&\multirow{8}{2.5cm}{i, ii, iii, iv, \(\cdots\)}\\\cline{1-2}
서문&필요시\\\cline{1-2}
사사&선택\\\cline{1-2}
목차&\\\cline{1-2}
표 목차&\multirow{2}{4cm}{\centering 본문에 표나 그림이 있는 경우}&\\\cline{1-1}
그림 목차&&\\\cline{1-2}
기호설명&선택\\\cline{1-2}\cline{4-4}
빈페이지&&&\\\cline{1-2}\cline{4-4}
본문&&&\multirow{4}{2.5cm}{1, 2, 3, 4, \(\cdots\)}\\\cline{1-2}
참고문헌&\\\cline{1-2}
부록&선택&\\\cline{1-2}
색인&선택&\\\hline
\end{tabular}
\caption{학위논문의 순서와 양식}
\label{tab:Organizing and formatting}
\end{table}

The paper size and margins are governed by the \text{geometry} package.
For more information, refer to the followings
\begin{itemize}
\item
\url{http://mirrors.ctan.org/macros/latex/contrib/geometry/geometry.pdf}
\item
\url{https://www.overleaf.com/learn/latex/Page_size_and_margins}
\end{itemize}

%%
\section{Fonts and Size}\label{sec:font}

본문의 글자 크기는 11pt로 설정되어 있다.

\LaTeX에서는 \verb|\normalsize|, \verb|\large|, \verb|\Large|, \verb|\LARGE|, \verb|\huge|와 같은 명령어를 사용해 글자크기를 변경할 수 있다.
이 명령어들은, MS word 템플릿에서 각각 11pt, 14pt, 16pt, 18pt and 21pt에 해당된다.
하지만, MS word 템플릿의 글자크기와는 약간의 차이가 있을 수 있다.
아래 표 \ref{tab:font size}는 글자 크기에 대한 비교표이다.
\footnote{\url{https://tug.org/texinfohtml/latex2e.html#Font-sizes}}
\begin{table}[h]\centering
\begin{tabular}{>{\centering\arraybackslash}p{6cm}cc}
\hline
&크기&\LaTeX 스타일\\\hline
논문제목				&21&\verb|\huge|\\\hline
학교이름(고려대학교)	&18&\verb|\LARGE|\\\hline
기타내용(학과명, 이름, 지도교수, \(\cdots\), 제출함, \(\cdots\), 완료함 등)
					&16&\verb|\Large|\\\hline
연 월				&14&\verb|\large|\\\hline
본문					&10--12&\verb|\normalsize|\\\hline
장, 절, 항 제목		&None&\\\hline
그림 제목			&None&\\\hline
표 제목				&None&\\\hline
\end{tabular}
\caption{글꼴 크기와 스타일 적용}\label{tab:font size}
\end{table}

%\begin{tabular}{ccc}
%MS word templete&\LaTeX commands&\LaTeX templete\\\hline
%11pt&\verb|\normalsize|&10.95pt\\
%14pt&\verb|\large|&12pt\\
%16pt&\verb|\Large|&14.4pt\\
%18pt&\verb|\LARGE|&17.28\\
%21pt&\verb|\huge|&20.74\\
%\end{tabular}
%\caption{Font sizes}\label{tab:font size}

방금 설명에서 표와 각주를 만들었다.
표를 만들기 위해서는 \texttt{tabular} 환경을 만들 수 있다.
이때, 열을 몇 개 사용할 것인지, 그리고 각 열들을 각각 어떻게 정렬할 것인지를 정해야 한다.
아래 코드는, 중앙정렬되어 있는 세 열을 통해 만드는 표의 예이다.
\begin{verbatim}
\begin{tabular}{ccc} ... \end{tabular}
\end{verbatim}
\texttt{tabular} 환경 외에도 which are \texttt{taubularx}, \texttt{tabulary}, \texttt{tabu}와 같은 환경 혹은 \texttt{multirow}, \texttt{booktabs}와 같은 패키지를 사용해 조금 더 정밀하게 표를 만들 수 있다.

\texttt{tabular} 환경은 \texttt{table} 환경 안에 넣는 것이 바람직하다.
표에 대한 캡션은 \verb|\caption{...}|와 같이 사용할 수 있고, 표에 대한 라벨 \verb|\label{...} |은 캡션이 선언된 직후에 입력되어야 한다.
\texttt{table} 환경에 포함된 모든 표들은 `표 목록'에 포함된다.

표를 조판하는 데 있어서 \url{https://www.overleaf.com/learn/latex/Tables}와 같은 자료를 참고할 수 있을 것이다.

%%
\section{그림과 수식}\label{sec:figures_and_equations}

그림은 본문 전체에 대해 연속적인 번호를 부여(1, 2, 3, 4, 5...) 하거나, 각 장(Chapter)에 기반하여 번호를 부여(1.2, 1.2, 2.1, 2.2...) 할 수 있다.
그림을 삽하기 위해서는 \texttt{includegraphics}와 같은 명령어를 사용할 수 있으며, 이 명령어를 사용하기 위해서는 \texttt{graphicx} 패키지가 필요하다.
\begin{verbatim}
\includegraphics[width=.2\textwidth]{kumark.png}
\end{verbatim}
위와 같은 명령에서, 대괄호 안에는 그림의 높이 혹은 너비를 특정할 수 있고, 중괄호 안에는 그림 파일의 파일명을 설정하여야 한다.
이때, 파일명은 확장자를 입력해도 좋고, 입력하지 않아도 좋다.

\texttt{includegraphics} 명령은 \texttt{figure} 환경 안에 넣는 것이 바람직하다.
이번에도, 캡션과 라벨을 넣을 수 있으며, 라벨은 캡션 직후에 선언되어야 한다.
\texttt{figure} 환경에 포함된 모든 그림들은 `그림 목록'에 포함된다.

\renewcommand\figurename{그림}
\begin{figure}[h]
\begin{center}
\includegraphics[width=.2\textwidth]{kumark.png}
\end{center}
\caption{고려대 심벌}
\label{fig:kumark}
\end{figure}

그림을 삽입하는 데 있어, 다음과 같은 자료들을 참고할 수 있다.
\begin{itemize}
\item
\url{https://www.overleaf.com/learn/latex/Inserting_Images}
\item
\url{https://www.overleaf.com/learn/latex/How_to_Write_a_Thesis_in_LaTeX_(Part_3)%3A_Figures%2C_Subfigures_and_Tables}
\end{itemize}

수식은 본문 전체에 대해 연속적인 번호를 부여((1), (2), (3), (4), (5)...) 하거나, 각 장(Chapter)에 기반하여 번호를 부여(1.2, 1.2, 2.1, 2.2...) 할 수 있다.

수식을 입력할 때에는 행중수식(inline mode)을 사용하여 \(E=mc^2\). % or $E=mc^2$
와 같이 입력할 수 있다.
혹은, 단락수식(display mode)을 사용해
\[E=mc^2\]
% or $$E=mc^2.$$
% or
%\begin{equation*}
%E=mc^2
%\end{equation*}
와 같이 사용할 수도 있다.
위의 단락수식은 식번호가 표시되지 않았는데, 단락수식에 식번호를 표시하기 위해서는 \texttt{equation} 환경을 사용할 수 있다.
\begin{equation}
E=mc^2
\end{equation}
위의 식번호는, 해당 수식이 첫번째 장의 첫번째 수식임을 나타내고 있다.
식을 하나 더 입력하면, 이것은 첫번째 장의 두번째 수식임을 표시할 것이다 ;
\begin{equation}
e^{i\theta}=\cos\theta+i\sin\theta.
\end{equation}
식번호를 개별적으로 지정할 수도 있다.
\verb|\tag{...}|를 사용하면 된다 ;
\[E=mc^2\tag{$*$}\]

여러 개의 식을 표현하려면, \texttt{gather} 환경을 사용할 수 있다.
\texttt{gather} 환경은 단순히 식들을 세로로 나열한다.
예를 들어, 연립방정식 \(x+y+z=3\), \(x-y+2z=1\), \(x+3z=2\)를 표현하기 위하여 \texttt{gather} 환경을 사용하면
\begin{gather}
x+y+z=3\\
x-y+2z=1\\
x+3z=2.
\end{gather}
와 같이 된다.
이렇게 하면, 각각의 식들에 대하여 식번호가 지정된다.
식번호를 지정하지 않으려면 \texttt{gather*} 환경을 사용할 수 있다.
\begin{gather*}
x+y+z=3\\
x-y+2z=1\\
x+3z=2.
\end{gather*}
지금까지는, 그냥 식들을 세로로 나열했을 뿐이다.
만약, 연립방정식의 식들을 잘 정렬시켜서 조판하려면 \texttt{align}/\texttt{align*} 환경을 사용하면 된다.
\texttt{align}/\texttt{align*} 환경을 사용할 때에는 `\&'을 통해 정렬시킬 대상을 지정해주면 된다.
아래의 예에서는 등호(=)를 기준으로 정렬한 것이다.
\begin{align*}
x+y+z&=3\\
x-y+2z&=1\\
x+3z&=2.
\end{align*}
\texttt{align} 환경을 사용하면 모든 식들에 식번호가 붙고, \texttt{align*} 환경을 사용하면 식번호가 전혀 붙지 않는다.
\begin{align}
x+y+z&=3\\
x-y+2z&=1\\
x+3z&=2.
\end{align}
연립방정식 전체를 하나의 식번호로 지정하려면 \texttt{aligned} 환경과 \texttt{equation} 환경을 동시에 사용하면 된다.
\begin{equation}\label{eq:system}
\begin{aligned}
x+y+z&=3\\
x-y+2z&=1\\
x+3z&=2.
\end{aligned}
\end{equation}
\verb|\label{...}|와 \verb|\eqref{...}|를 활용하면 식들에 대해서도 상호참조를 할 수 있다.
예를 들어, 다음과 같이 쓸 수 있다.
`식 \eqref{eq:system}의 근은 \(x=2\), \(y=1\), \(z=0\)이다.'

위와 같은 수식 환경들 (\texttt{gather}, \texttt{align})은 \texttt{amsmath} 패키지를 통해 제공된다.
수식을 적절하게 조판하기 위해서는 다음과 같은 자료들을 참고할 수 있을 것이다.
\begin{itemize}
\item
\url{http://www.ams.org/arc/tex/amsmath/amsldoc.pdf}
\item
\url{http://wiki.ktug.org/wiki/wiki.php/수식작성}
\end{itemize}

%%
\section{인용}
인용을 위해서는 \texttt{quotation} 환경 혹은 \text{quote} 환경을 사용할 수 있다.

%%
\section{각주}
각주를 사용하기 위해서는 \verb|\footnote{...}|와 같은 명령어를 사용할 수 있다.

%%% the third chapter of the main body
\chapter{분석}\label{chap:discussion}
Discussion starts here.

논문에 정의나 정리 등을 넣기 위해서는, 해당 환경들을 직접 정의해 사용할 수 있다.
전처리부분(preamble, \verb|\begin{document}|의 앞부분)에 보면 \texttt{definition} 환경과 \texttt{theorem} 환경을 각각 정의해놓았었다.

\texttt{definition} 환경을 사용할 때에는, 다음과 같이 정의의 이름을 특정해주어도 되고
\begin{definition}[직각삼각형]
직각삼각형은 한 각이 직각인 삼각형이다. 
\end{definition}
특정하지 않아도 된다.
\begin{definition}
직각삼각형은 한 각이 직각인 삼각형이다. 
\end{definition}

\texttt{theorem} 환경도 마찬가지로 사용해줄 수 있다 ;
\begin{theorem}[피타고라스 정리]
밑변의 길이가 \(a\), 높이가 \(b\), 빗변의 길이가 \(c\)인 직각삼각형에 대하여, 다음 식이 성립한다.
\begin{equation}
a^2+b^2=c^2
\end{equation}
\end{theorem}

\begin{theorem}
밑변의 길이가 \(a\), 높이가 \(b\), 빗변의 길이가 \(c\)인 직각삼각형에 대하여, 다음 식이 성립한다.
\begin{equation}
a^2+b^2=c^2
\end{equation}
\end{theorem}

수식을 입력할 때에는, 가끔 특별한 종류의 알파벳을 사용해야 하는 경우가 있다.
예를 들어, \(\mathbb R\), \(\mathcal T\), \(\mathscr A\), \(\mathfrak M\)와 같은 종류의 알파벳들을 사용할 수 있다.
이러한 특수한 종류의 알파벳들 중 기본적으로 지원되는 것도 있지만, 몇몇 종류들은 \text{amssymb} or \text{mathrsfs}와 같은 패키지가 필요할 수도 있다.
수식 입력에 관한 더 많은 정보를 위해서는 다음 자료들을 참고할 수 있다 ;
\begin{itemize}
\item
\url{https://www.overleaf.com/learn/latex/Mathematical_expressions}
\item
\url{https://www.overleaf.com/learn/latex/Subscripts_and_superscripts}
\item
\url{https://www.overleaf.com/learn/latex/Brackets_and_Parentheses}
\item
\url{https://www.overleaf.com/learn/latex/Matrices}
\item
\url{https://www.overleaf.com/learn/latex/Integrals\%2C_sums_and_limits}
\item
\url{https://www.overleaf.com/learn/latex/Display_style_in_math_mode}
\item
\url{https://www.overleaf.com/learn/latex/Mathematical_fonts}
\end{itemize}

나중에 색인에서 쓰기 위하여 직각삼각형\index{right traingle}과 피타고라스정리\index{pythagorean theorem}에 대한 인덱싱을 여기에 해놓겠다.
색인 장(chapter)에 \verb|\printindex| 명령어를 입력하기만 하면, 기존에 지정해놓은 모든 색인들이 자동적으로 나타난다.

%%% the fourth chapter of the main body
\chapter{결론}\label{chap:conclusion}
Conclusion starts here.

참고문헌에 수록되어 있는 문헌들을 인용하기 위해서는 \verb|\cite{LSTM}|와 같은 명령어를 사용할 수 있다.
이 명령어를 사용하고 나면 \cite{LSTM}와 같이 잘 인용되는 것을 확인할 수 있다.
이때, \texttt{LSTM}는 특정 논문에 대하여 지정해놓은 이름이다.
또다른 참고문헌에 대해서는 \cite{pure}와 같이 인용할 수 있을 것이다.

%%% the bibliography chapter
\renewcommand\bibname{참고문헌(서지)}
\addcontentsline{toc}{chapter}{참고문헌(서지)}
\begin{thebibliography}{AA}
\bibitem {LSTM} Hochreiter, Sepp, and Jürgen Schmidhuber. ``Long short-term memory.'' Neural computation 9.8 (1997): 1735-1780.
\bibitem {pure} Hardy, Godfrey Harold. Course of pure mathematics. Courier Dover Publications, 2018.
\end{thebibliography}
%Reference starts here.
%
%References are a detailed list of sources that are cited in your thesis/dissertation. A bibliography is a detailed list of references cited in your thesis/dissertation plus background or other material you have read but have not actually cited.
%
%References should be prepared in a consistent format using bibliographic management tools (Endnote, Mendeley, etc.) in the order of author name or citation according to your academic field.
%
%Bibliographic management tools
%\begin{itemize}
%\item\url{https://library.korea.ac.kr/research/writing-guide/endnote/}
%\item\url{https://library.korea.ac.kr/research/writing-guide/mendeley/}
%\end{itemize}

%%% the appendix chapters
\appendix
\addcontentsline{toc}{chapter}{부록}
\chapter{The first  appendix}
A text for appendix 1 starts here.
\chapter{The second appendix}
A text for appendix 2 starts here.

%%% the index chapter
\addcontentsline{toc}{chapter}{색인}
\printindex

\end{document}