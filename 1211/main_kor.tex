\documentclass[11pt]{report}
%\usepackage{amscd,epsfig,longtable}
%\usepackage[T1]{fontenc}
\usepackage{amsthm, amsmath, amssymb, mathrsfs}
\usepackage{geometry, graphicx, kotex, tabularx, verbatim, setspace, url}
\usepackage[table]{xcolor}
\usepackage{lipsum}
\usepackage{ragged2e}
\usepackage{indentfirst}
\geometry{paper=b5paper, left=30mm, right=30mm, top=30mm, bottom=30mm}

%\numberwithin{figure}{chapter}
\numberwithin{figure}{section}

\theoremstyle{plain}
\newtheorem{theorem}{\protect\theoremname}[chapter]
\theoremstyle{definition}
\newtheorem{definition}[theorem]{\protect\definitionname}
\theoremstyle{corollary}
\newtheorem{corollary}[theorem]{\protect\corollaryname}
\theoremstyle{definition}
\newtheorem{rem}[theorem]{\protect\remarkname}
\theoremstyle{plain}
\newtheorem{proposition}[theorem]{\protect\propositionname}
\theoremstyle{definition}
\newtheorem{claim}[theorem]{\protect\examplename}
\theoremstyle{plain}
\newtheorem{lemma}[theorem]{\protect\lemmaname}
\newtheorem{assumption}[theorem]{Assumption}

\renewcommand{\labelenumi}{(\roman{enumi})}

\providecommand{\lemmaname}{Lemma}
\providecommand{\propositionname}{Proposition}
\providecommand{\remarkname}{Remark}
\providecommand{\theoremname}{Theorem}
\providecommand{\examplename}{Claim}
\providecommand{\definitionname}{Definition}
\providecommand{\corollaryname}{Corollary}

\begin{document}
\onehalfspacing
\renewcommand{\arraystretch}{1.5}

\newpage
\centering
\pagenumbering{gobble} % Cover page, title page and signature page) should not be numbered.
\Large 석(박) 사 학 위 논 문
\par\vspace{3cm} % 3cm spacing
\huge 학위 논문 제목
\par\vspace{0.3cm}\Large - 부제가 있을 경우 중앙에 위치
\par\vspace{4.7cm} % spacing can be adjusted
\LARGE 고 려 대 학 교 ~ 대 학 원
\par\vspace{0.5cm}
\Large OOO 학과
\par\vspace{0.5cm}
\Large 홍 길 동
\par\vspace{3cm}
\large 2023년 2월

\newpage
~
% Blank page should be inserted after the cover page.


\newpage
\Large 김 철 수 ~ 교 수 지 도
\par\vspace{0.5cm}
\Large 석(박) 사 학 위 논 문
\par\vspace{2cm}
\huge 학위 논문 제목
\par\vspace{0.3cm}\Large - 부제가 있을 경우 중앙에 위치
\par\vspace{2cm} % spacing can be adjusted
\Large 이 논문을 O학 석(박)사학위 논문으로 제출함
\par\vspace{2cm}
\large 2022년 10월
\par\vspace{2cm}
\LARGE 고 려 대 학 원 ~ 대 학 원
\par\vspace{0.5cm}
\Large OOO 학과
\par\vspace{1cm}
\Large 홍길동
\par\vspace{1.0cm}
\rule{.6\textwidth}{0.4pt} % Student's signature is required on the line

\newpage % For Master's Degree only
\par\vspace{1cm}
\Large 홍길동의 O학 석(박)사학위논문 심사를 완료함
\par\vspace{2.5cm}
\Large 20OO년 O월
\par\vspace{2cm}
\rule{.6\textwidth}{0.4pt}\par % committee's signature above the line 
\Large 위원장 : OOO 교수
\par\vspace{1cm}
\rule{.6\textwidth}{0.4pt}\par % committee's signature above the line 
위원 : OOO 교수
\par\vspace{1cm}
\rule{.6\textwidth}{0.4pt}\par % committee's signature above the line 
위원 : OOO 교수
\par\vspace{1cm}
\rule{.6\textwidth}{0.4pt}\par % committee's signature above the line 
위원 : OOO 박사
\par\vspace{1cm}
\rule{.6\textwidth}{0.4pt}\par % committee's signature above the line 
위원 : OOO 박사

\newpage
\newgeometry{paper=b5paper, left=20mm, right=20mm, top=30mm, bottom=30mm}
% From the abstract on, the top and bottom margins shall be at least 3cm and the left and right margins shall be at least 2 cm.
\pagenumbering{roman}
%페이지 번호는 초록부터 본문 전까지 작은 로마 숫자(Roman numerals, e.g., i, ii, iii, iv...)를 사용한다.
\begin{center}
\LARGE 국문 제목
\par\vspace{20pt}
\normalsize 홍 길 동\par
O O O 학 과\par
지도교수: 김 철 수

\par\vspace{20pt}
\addcontentsline{toc}{section}{국문초록}
\section*{국문초록}

\end{center}
\normalsize
국문 학위논문의 초록은 국문, 영문의 순서로 작성하며, 영문 학위논문의 초록은 영문, 국문의 순서로 작성하며, 학위논문을 기타 외국어로 작성하는 경우 초록은 기타 외국어, 영문, 국문의 순서로 작성한다.

초록에는 논문제목, 성명, 학과, 지도교수를 기재하며 초록 하단에 주요어(keywords)를 표기한다. 

\par\vspace{100pt}

\textbf{중심어} : 중심어, 중심어, 중심어, 중심어, 중심어, 중심어

\newpage
\begin{center}
\LARGE Title
\par\vspace{20pt}
\doublespacing
\normalsize by Gildong Hong\par
Department of OOOO\par
under the supervision of Professor Chulsu Kim

\par\vspace{20pt}
\addcontentsline{toc}{section}{Abstract}
\section*{Abstract}
\end{center}

\justifying % this command available with \usepackage{ragged2e}
\doublespacing
\normalsize
The text of the abstract begins here. 
\par\vspace{10pt}

\textbf{Keywords} : Keyword, Keyword, Keyword, Keyword, Keyword, Keyword

\newpage
~
\vspace{5.5cm} \par
\begin{center}
감사의 글은 필요한 경우 작성한다.\par
감사의 글 작성시, 페이지 중앙에 위치하도록 한다.\par
감사의 글(Dedication) 제목은 생략하여 목차에 표현하지 않는 것이 일반적이다.
\end{center}

\newpage
\begin{center}
\addcontentsline{toc}{section}{서문}
\section*{서문}
\end{center}

\normalsize
학위논문에 다른 사람들과 협력하여 수행된 결과가 포함되거나, 저자가 출판한 내용이 포함되는 경우, 이와 관련된 내용을 서문에 작성하여야 한다. 서문에는 아래의 내용이 포함될 수 있다. 다만, 서문을 따로 작성하지 않고, 관련 사항을 본문의 서론에서 언급하는 것도 가능하다. \par

① 다른 사람들과 협력하여 수행한 작업에 대한 다른 사람의 기여도와 비율 및 저자가 독창적이라고 주장하는 부분에 대한 설명\par
② 논문의 일부분이 이미 출판되었거나 준비 중인 부분에 대한 설명 및 출판물에 대한 모든 사람의 기여\par
③ 이외에도 논문작성 관련 개인적 상황 및 정보(personal information), 주제 선택 동기(motivation), 저자 관점, 감사 및 사사(acknowledgments) 등의 내용이 포함될 수 있다.\par

\bigskip
서문 작성 예
\begin{itemize}
\item\url{https://www.grad.ubc.ca/sites/default/files/doc/page/thesis_sample_prefaces.pdf}
\item\url{https://www.phase-trans.msm.cam.ac.uk/2002/thomas/chapter1.pdf}
\end{itemize}

\newpage
\begin{center}
\addcontentsline{toc}{section}{사사}
\section*{사사}
\end{center}

\normalsize
필요한 경우 사사를 작성한다. \par
서문(Preface)에서 사사(acknowledgments)와 관련된 내용을 기술한 경우, 생략할 수 있다.

\newpage
\renewcommand*\contentsname{목차}
\addcontentsline{toc}{section}{목차}
\tableofcontents

% 위의 목차는 모든 장, 절, 항 제목에 대해 자동적으로 생성된다.
% 페이지 번호는 초록부터 본문 전까지 작은 로마 숫자(Roman numerals, e.g., i, ii, iii, iv...)를 사용한다. 본문의 서론부터 아라비아 숫자(Arabic numbers, e.g., 1, 2, 3...)를 사용한다.
% 본문의 장은 아라비아 숫자(1, 2, 3, 4...), 부록은 영문 알파벳(A, B, C...)을 사용하여 구분하는 것이 일반적이다.


\renewcommand{\listtablename}{표 목차}
\addcontentsline{toc}{section}{표 목차}
\listoftables

% 위의 표 목차는 table 환경의 모든 표 제목에 대해 자동적으로 생성된다.
% 본문에 표가 포함되는 경우 반드시 표 목차를 작성한다. 본문을 한글로 작성하더라도 표 제목는 영어로 작성 가능하다. 표는 본문 전체에 대해 연속적인 번호를 부여(1, 2, 3, 4, 5...) 하거나, 각 장(Chapter)에 기반하여 번호를 부여(1.2, 1.2, 2.1, 2.2...) 할 수 있다.

\renewcommand{\listfigurename}{그림 목차}
\addcontentsline{toc}{section}{그림 목차}
\listoffigures

% 위의 그림 목차는 figure 환경의 모든 그림 제목에 대해 자동적으로 생성된다.
% 본문에 그림이 포함되는 경우 반드시 그림 목차를 작성한다. 본문을 한글로 작성한 경우에도 그림 제목는 영어로 작성 가능하다. 그림은 본문 전체에 대해 연속적인 번호를 부여(1, 2, 3, 4, 5...) 하거나, 각 장(Chapter)에 기반하여 번호를 부여(1.2, 1.2, 2.1, 2.2...) 할 수 있다.

\newpage
\begin{center}
\addcontentsline{toc}{section}{Nomenclature(or List of Symbols)}
\section*{Nomenclature(or List of Symbols)}
\end{center}
\normalsize
\begin{tabular}{p{.2\textwidth}p{.7\textwidth}}
$M$	& original mass matrix\\
$K$	& original stiffness matrix\\[30pt]
\multicolumn{2}{l}{Subscripts}\\
$b$ & interface boundary\\
$d$ & dominant\\[30pt]
\multicolumn{2}{l}{Abbreviation}\\
$CMS$ & Component Mode Synthesis\\
\end{tabular}
%필요한 경우 기호설명을 작성한다. 기호 설명에는 필요한 경우 첨자 설명 및 약어 설명을 포함한다. 본문을 한글로 작성한 경우에도 기호 설명은 영어로 작성 가능하다.

\newpage 
\ % 본문 앞에는 빈페이지를 둔다. 
\pagenumbering{gobble} % Blank page should not be numbered.

%%%
\chapter{서론}\label{chap:intro}
\pagenumbering{arabic}

장(chapter)을 만들기 위해 \verb|\chapter{서론}|을 사용하였다.
이 장을 라벨링 하기 위해서는 \verb|\label{chap:intro}|와 같은 명령어를 사용할 수 있다.

%%
\section{절 제목}\label{sec:section}
절(section)을 만들기 위해 \verb|\section{절 제목}|을 사용하였다.
이 절을 라벨링 하기 위해서는 \verb|\label{sec:section}|와 같은 명령어를 사용할 수 있다.
이 템플릿에서 제공하는 장, 절, 항의 양식은 하나의 예시일 뿐이다.
따라서 이 양식을 꼭 따라야 할 필요는 없다.

%
\subsection{항 제목}\label{subs:subsection}
항(subsection)을 만들기 위해 \verb|\subsection{항 제목}|을 사용였다.
이 항을 라벨링 하기 위해서는 \verb|\label{subs:subsection}|와 같은 명령어를 사용할 수 있다.

장, 절, 항들은 목차에 자동적으로 표시된다.

%%%
\chapter{학위논문의 양식}\label{chap:organizing}

%%
\section{용지 크기, 여백 및 페이지 설정} \label{sec:papersize}
논문의 규격은 4·6배판(B5)로 하는 것을 원칙으로 한다.

논문 표지, 속표지, 심사완료검인서의 아래쪽, 위쪽, 오른쪽, 왼쪽의 여백은 3cm 이상으로 한다. 초록부터 페이지 여백은 아래쪽, 위쪽, 3cm 이상 오른쪽, 왼쪽 2cm 이상으로 한다.

페이지 번호는 초록부터 본문 전까지 작은 로마 숫자(Roman numerals, e.g., i, ii, iii, iv...)를 사용하며, 본문의 서론부터 아라비아 숫자(Arabic numbers, e.g., 1, 2 , 3...)를 사용한다.

표는 본문 전체에 대해 연속적인 번호를 부여(1, 2, 3, 4, 5...) 하거나, 각 장(Chapter)에 기반하여 번호를 부여(1.2, 1.2, 2.1, 2.2...) 할 수 있다. (Table \ref{tab:Organizing and formatting}).

\renewcommand\tablename{표}
\begin{table}
\caption{학위논문의 순서와 양식}
\label{tab:Organizing and formatting}
\vspace{0.5cm}
\begin{tabular}{ m{7cm} m{3cm} m{2cm}}
\hline
순서(비고) & 여백설정& 페이지설정\\\hline
논문표지, 빈페이지, 속표지, 심사완료검인서  &	 위, 아래, 왼쪽 오른쪽 모두 3 cm 이상	&	없음\\\hline
국문 및 영문 초록 , 서문(필요시), 사사(선택), 목차, 표 목차 (본문에 표가 있는 경우), 그림 목차 (본문에 그림이 있는 경우), 기호설명(선택)	& 위, 아래 3cm 이상, 왼쪽, 오른쪽 2cm 이상  &  i, ii, iii...         \\\hline		
빈 페이지 & 위와 같음 & 없음\\\hline
본문, 참고문헌, 부록(선택), 색인(선택) & 위와 같음 & 1,2,3...	\\\hline

\end{tabular}
\end{table}

%%

\begin{table}\centering
\caption{Fonts setting used in this document}
\vspace{0.5cm}
\begin{tabular}{  m{7cm}  m{5cm} }
\hline 				&     \LaTeX{} Command \\\hline 
논문제목				& huge \\
학교이름(고려대학교)	& LARGE \\
연, 월, 				& large\\
기타 내용 (학과명, 이름, 지도교수, \(\cdots\), 제출함, \(\cdots\), 완료함,등)	& Large\\
본문					& normalsize	\\
장 제목   			& chapter \\
절 제목				& section \\
항 제목				& subsection \\\hline

\end{tabular}
\end{table}



%%
\newpage
\section{Figures and Equations}\label{sec:figures_and_equations}

그림을 삽하기 위해서는 \texttt{includegraphics}와 같은 명령어를 사용할 수 있으며, 이 명령어를 사용하기 위해서는 \texttt{graphicx} 패키지가 필요하다.
\texttt{includegraphics} 명령은 \texttt{figure} 환경 안에 넣는 것이 바람직하다.
\texttt{figure} 환경에 포함된 모든 그림들은 `그림 목록'에 포함된다.

\renewcommand\figurename{그림}
\begin{figure}
\begin{center}
\includegraphics[width=.2\textwidth]{kumark.png}
\end{center}
\caption{고려대 심벌}
\end{figure}

\begin{equation}
E=mc^2
\end{equation}
위의 식번호는, 해당 수식이 두번째 장의 첫번째 수식임을 나타내고 있다.
식을 하나 더 입력하면, 이것은 두번째 장의 두번째 수식임을 표시할 것이다 ;
\begin{equation}
e^{i\theta}=\cos\theta+i\sin\theta.
\end{equation}

여러 개의 수식을 입력할 때, 각각의 수식들에 대해 식번호를 부여할 수도 있고
\begin{align}
x+y+z&=3\\
x-y+2z&=1\\
x+3z&=2
\end{align}
아니면, 연립방정식 전체에 대하여 식번호를 하나 부여할 수도 있다.
\begin{equation}
\begin{aligned}
x+y+z&=3\\
x-y+2z&=1\\
x+3z&=2
\end{aligned}
\end{equation}

\section{Footnotes and Endnotes}\label{sec:footnotes_endnotes}

각주는 논문 내용에 대한 부가적인 설명을 제공하기 위하여 삽입할 수 있다.
수평선을 경계로 본문의 밑에 분리되어 있으며, 해당 단어에 달린 각주번호를 통해 주석의 내용을 참조할 수 있도록 되어 있다.

\section{직접인용}\label{sec:quotation}
저자의 의도를 독자에게 효율적으로 전달하기 위해서는, 가끔 직접인용이 필요한 경우가 있다.\par
\bigskip

\leftskip 1.5cm
\rightskip 1.5cm 

“Learn from yesterday, live for today, hope for tomorrow. The important thing is not to stop questioning.” the universe.” \par
― Albert Einstein 

\leftskip 0cm
\rightskip 0cm 

%%%
\chapter{논의}\label{chap:discussion}
The discussion starts here.

%%%
\chapter{결론}\label{chap:conclusion}
The conclusion starts here. \par



\begin{center}
\addcontentsline{toc}{chapter}{참고문헌} % or Bibliography
\chapter*{참고문헌} % or Bibliography
\end{center}

\normalsize
본문 뒤에는 참고문헌(References) 또는 서지(Bibliography)를 작성한다.\par
참고문헌(References)은 본문에서 인용되거나 참고한 자료를 작성한 목록을 말한다. 서지(Bibliography)는 엄밀한 의미에서 참고문헌(References) 뿐만 아니라 음반, 면담, 영화, TV프로그램, 그림 등 비문자 자료를 모두 포괄한다. \par
참고문헌은 서지관리 프로그램(Endnote, Mendeley 등)을 사용하여 학문분야 특성에 맞게 저자 이름순 혹은 인용순 등으로 일관된 양식으로 작성한다.


\bigskip

서지관리 프로그램 링크
\begin{itemize}
\item\url{https://library.korea.ac.kr/research/writing-guide/endnote/}
\item\url{https://library.korea.ac.kr/research/writing-guide/mendeley/}
\end{itemize}

\begin{center}
\addcontentsline{toc}{chapter}{부록} % or Supplementary Materials
\chapter*{부록} % or Supplementary Materials
\end{center}

\normalsize
\addcontentsline{toc}{section}{A. 부록 제목} 
\section*{A. 부록 제목} % or Supplementary Materials
필요한 경우 부록(appendices or supplementary materials)을 작성한다.
부록의 각 장은 영문 알파벳을 사용하여 구분하는 것이 일반적이다.

\begin{center}
\addcontentsline{toc}{chapter}{색인}
\chapter*{색인}

\end{center}
\normalsize
필요한 경우 색인(index)을 작성한다.


\end{document}

