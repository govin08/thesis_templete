\documentclass[11pt]{report}
%\usepackage{amscd,epsfig,longtable}
%\usepackage[T1]{fontenc}
\usepackage{amsthm, amsmath, amssymb, mathrsfs}
\usepackage{graphicx, kotex, tabularx, multirow, setspace, url}
\usepackage[table]{xcolor}
\usepackage{geometry}
\geometry{paper=b5paper, left=30mm, right=30mm, top=30mm, bottom=30mm}
\counterwithin{table}{chapter}

\theoremstyle{plain}
\newtheorem{theorem}{\protect\theoremname}[chapter]
\theoremstyle{definition}
\newtheorem{definition}[theorem]{\protect\definitionname}
\theoremstyle{corollary}
\newtheorem{corollary}[theorem]{\protect\corollaryname}
\theoremstyle{definition}
\newtheorem{rem}[theorem]{\protect\remarkname}
\theoremstyle{plain}
\newtheorem{proposition}[theorem]{\protect\propositionname}
\theoremstyle{definition}
\newtheorem{claim}[theorem]{\protect\examplename}
\theoremstyle{plain}
\newtheorem{lemma}[theorem]{\protect\lemmaname}
\newtheorem{assumption}[theorem]{Assumption}

\renewcommand{\labelenumi}{(\roman{enumi})}

\providecommand{\lemmaname}{Lemma}
\providecommand{\propositionname}{Proposition}
\providecommand{\remarkname}{Remark}
\providecommand{\theoremname}{Theorem}
\providecommand{\examplename}{Claim}
\providecommand{\definitionname}{Definition}
\providecommand{\corollaryname}{Corollary}

\renewcommand{\theequation}{\thesection.\arabic{equation}}

\begin{document}
\onehalfspacing
\renewcommand{\arraystretch}{1.5}

\newpage
\noindent
\begin{tabularx}{\textwidth}{| >{\centering\arraybackslash}X |}
\arrayrulecolor{gray}
\hline
\Large 석(박) 사 학 위 논 문 \\\hline
~\small\color{gray}{3cm spacing}\vspace{70pt}\\\hline % 0.5cm is approximately 14pt, where default vertical spacing is 0.5cm(14pt). %5\times14pt=70pt
\huge 학위 논문 제목\\
\Large 부제가 있을 경우 중앙에 위치\\
~\small\color{gray}{여백 조정 가능}\vspace{14pt}\\\hline 
\LARGE 고 려 대 학 교 ~ 대 학 원\\\hline % ~ for spacing
~\small\color{gray}{0.5cm 여백}\\\hline
\Large OOO 학과 \\\hline
~\small\color{gray}{0.5cm 여백}\\\hline
\Large 홍 길 동 \\\hline
~\small\color{gray}{3cm spacing}\vspace{70pt}\\\hline %5\times14pt=70pt
\large 2023년 2월 \\\hline
\end{tabularx}

\newpage
~
%논문 표지 뒤에는 빈 페이지를 삽입합니다.
%
%본 양식은 대학원 학칙 일반대학원 시행세칙 및 일반대학원 학위논문 작성법에 따라서 MS Word를  사용하여 2022년 10월 작성되었다. 
%
%- 22년 10월 21일 업데이트 사항 – 
%직접 인용 예시 제공 / 각주 스타일 추가 제공
%
%한국출판협회에 무료로 배포하는 KoPub 2.0 서체와 Times New Roman 서체를 사용하였다. 
%
%https://www.kopus.org/biz-electronic-font2-2/
%
%제공된  양식에서 논문의 순서 및 겉표지, 속표지, 심사완료검인서, 초록 페이지에 안내된 내용은 반드시 따라야 하는 부분이며, 나머지 부분들은 한가지 예시로 제공되었으므로, 수정하여 사용 가능하다. 예로 장, 절, 항 제목의 서체, 글자크기, 정렬방식은 적절하게 변경하여 사용할 수 있다. 나머지 부분에서 반드시 지켜야 할 내용, 예를 들어 본문 글자 사이즈(10-12 포인트), 페이지 매기기 등은 “일반대학원_학위논문_작성법” 파일을 참고하여야 한다.
%또한, 제공된 양식에서 설명 글에 해당하는 부분은 삭제하여야 하며, 겉표지, 속표지 등에 있는 테이블 줄은 보이지 않게 처리하여야 한다.

\newpage
\noindent
\begin{tabularx}{\textwidth}{| >{\centering\arraybackslash}X |}
\arrayrulecolor{gray}
\hline
\Large 김 철 수 ~ 교 수 지 도 \\\hline
~\small\color{gray}{0.5cm 여백}\\\hline
\Large 석(박) 사 학 위 논 문 \\\hline
~\small\color{gray}{2-3cm 여백}\vspace{35pt}\\\hline %3\times14pt=42pt
\huge 학위 논문 제목\\
\Large 부제가 있을 경우 중앙에 위치\\
~\small\color{gray}{여백 조정 가능}\vspace{28pt}\\\hline 
\Large 이 논문을 O학 석(박)사학위 논문으로 제출함\\\hline
~\small\color{gray}{2-3cm 여백}\vspace{35pt}\\\hline %3\times14pt=42pt
\large 2023년 2월\\\hline
~\small\color{gray}{2-3cm 여백}\vspace{35pt}\\\hline %3\times14pt=42pt
\LARGE 고 려 대 학 교 ~ 대 학 원\\\hline % ~ for spacing
~\small\color{gray}{0.5cm 여백}\\\hline
\Large OOO 학과 \\\hline
~\small\color{gray}{1cm 여백}\vspace{14pt}\\\hline %1\times14pt=14pt
\Large 홍길동 (인) \\\hline
\end{tabularx}

\newpage
\noindent
\begin{tabularx}{\textwidth}{| >{\centering\arraybackslash}X |}
\arrayrulecolor{gray}
\hline
~\small\color{gray}{1cm 여백}\vspace{14pt}\\\hline %1\times14pt=14pt
\Large 홍길동의 O학 석(박)사학위 논문 심사를 완료함\\\hline
~\small\color{gray}{2-3cm 여백}\vspace{42pt}\\\hline %3\times14pt=42pt
\Large 20OO년 O월\\\hline
~\small\color{gray}{2cm 여백}\vspace{42pt}\\\hline %3\times14pt=42pt
\Large 위\phantom{원}원\qquad\phantom{박사의 경우 추가 }\qquad (인)\\\hline
~\small\color{gray}{0.5cm 여백}\vspace{14pt}\\\hline
\Large 위\phantom{원}원\qquad\phantom{박사의 경우 추가 }\qquad (인)\\\hline
~\small\color{gray}{0.5cm 여백}\vspace{14pt}\\\hline
\Large 위\phantom{원}원\qquad\phantom{박사의 경우 추가 }\qquad (인)\\\hline
~\small\color{gray}{0.5cm 여백}\vspace{14pt}\\\hline
\Large 위\phantom{원}원\qquad박사의 경우 추가 \qquad (인)\\\hline
~\small\color{gray}{0.5cm 여백}\vspace{14pt}\\\hline
\Large 위\phantom{원}원\qquad박사의 경우 추가 \qquad (인)\\\hline
~\small\color{gray}{0.5cm 여백}\vspace{14pt}\\\hline
\end{tabularx}

\newpage
\newgeometry{paper=b5paper, left=20mm, right=20mm, top=30mm, bottom=30mm}
% 이 페이지부터 여백(아래쪽, 위쪽, 3cm, 오른쪽, 왼쪽 2cm) 변경됨
% Margins shall be changed to bottom and top 3 cm, right and left 2 cm from this page forward.

\newpage
\begin{center}
\LARGE 국문 제목
\par\vspace{20pt}
\normalsize by 홍길동\par
OO 학과\par
지도교수 : 김철수

\par\vspace{20pt}

\large \textbf{국문 초록}
\end{center}
\normalsize
국문 학위논문의 초록은 국문, 영문의 순서로 작성하며, 영문 학위논문의 초록은 영문, 국문의 순서로 작성하며, 학위논문을 기타 외국어로 작성하는 경우 초록은 기타 외국어, 영문, 국문의 순서로 작성한다. 

초록에는 논문제목, 성명, 학과, 지도교수를 기재하며 초록 하단에 주요어(keywords)를 표기한다. 
페이지 번호는 초록부터 본문 전까지 작은 로마 숫자(Roman numerals, e.g., i, ii, iii, iv...)를 사용한다.

\par\vspace{100pt}

\textbf{중심어} : 중심어, 중심어, 중심어, 중심어, 중심어, 중심어


\newpage
\begin{center}
\LARGE Title
\par\vspace{20pt}
\normalsize by Gildong Hong\par
Department of OOOO\par
under the supervision of Professor Chulsu Kim

\par\vspace{20pt}

\large \textbf{ABSTRACT}
\end{center}
\normalsize
The text of the abstract begins here. 

The above title line (ABSTRACT) is styled using \verb|\large| and  \verb|\textbf|.

Paragraph text is styled using default style.

Pages should be assigned from the abstract using small Roman numericals (i, ii, iii, iv, v, etc.)

\par\vspace{100pt}

\textbf{Keywords} : Keyword, Keyword, Keyword, Keyword, Keyword, Keyword

\newpage
~ % an empty page

\newpage
~
%감사의 글은 필요한 경우 작성한다. 감사의 글 작성시, 페이지 중앙에 위치하도록 한다. 감사의 글(Dedication) 제목은 생략하여 목차에 표현하지 않는 것이 일반적이다.

\newpage
\begin{center}
\large
Preface
\end{center}
\normalsize
The text of the preface begins here. 

%학위논문에 다른 사람들과 협력하여 수행된 결과가 포함되거나, 저자가 출판한 내용이 포함되는 경우, 이와 관련된 내용을 서문에 작성하여야 한다. 서문에는 아래의 내용이 포함될 수 있다. 다만, 서문을 따로 작성하지 않고, 관련 사항을 본문의 서론에서 언급하는 것도 가능하다. 
%① 다른 사람들과 협력하여 수행한 작업에 대한 다른 사람의 기여도와 비율 및 저자가 독창적이라고 주장하는 부분에 대한 설명
%② 논문의 일부분이 이미 출판되었거나 준비 중인 부분에 대한 설명 및 출판물에 대한 모든 사람의 기여
%③ 이외에도 논문작성 관련 개인적 상황 및 정보(personal information), 주제 선택 동기(motivation), 저자 관점, 감사 및 사사(acknowledgments) 등의 내용이 포함될 수 있다.

%
%\bigskip
%서문 작성 예
%\begin{itemize}
%\item\url{https://www.grad.ubc.ca/sites/default/files/doc/page/thesis_sample_prefaces.pdf}
%\item\url{https://www.phase-trans.msm.cam.ac.uk/2002/thomas/chapter1.pdf}
%\end{itemize}

\newpage
\begin{center}
\large
사사
\end{center}
\normalsize
필요한 경우 사사를 작성한다. 

서문(Preface)에서 사사(acknowledgments)와 관련된 내용을 기술한 경우, 생략할 수 있다.

\renewcommand{\contentsname}{목차}
\tableofcontents
%위의 목차는 스타일 “제목1, 2, 3”이 적용된 장, 절, 항 제목에 대해 자동적으로 생성된다.
%목차는 초록부터 작성한다.
%페이지 번호는 초록부터 본문 전까지 작은 로마 숫자(Roman numerals, e.g., i, ii, iii, iv...)를 사용한다. 본문의 서론부터 아라비아 숫자(Arabic numbers, e.g., 1, 2, 3...)를 사용한다.
%본문의 장은 아라비아 숫자(1, 2, 3, 4...), 부록은 영문 알파벳(A, B, C...)을 사용하여 구분하는 것이 일반적이다.

\renewcommand{\listtablename}{표 목차}
\listoftables
%위의 표 목차는 스타일 “Table Title”이 적용된 표 제목에 대해 자동적으로 생성된다.
%본문에 표가 포함되는 경우 반드시 표 목차를 작성한다. 본문을 한글로 작성하더라도 표 제목는 영어로 작성 가능하다. 표는 본문 전체에 대해 연속적인 번호를 부여(1, 2, 3, 4, 5...) 하거나, 각 장(Chapter)에 기반하여 번호를 부여(1.2, 1.2, 2.1, 2.2...) 할 수 있다.

\renewcommand{\listfigurename}{그림 목차}
\listoffigures
%위의 그림 목차는 스타일 “Table Title”이 적용된 그림 제목에 대해 자동적으로 생성된다.
%본문에 그림이 포함되는 경우 반드시 그림 목차를 작성한다. 본문을 한글로 작성한 경우에도 그림 제목는 영어로 작성 가능하다. 그림은 본문 전체에 대해 연속적인 번호를 부여(1, 2, 3, 4, 5...) 하거나, 각 장(Chapter)에 기반하여 번호를 부여(1.2, 1.2, 2.1, 2.2...) 할 수 있다.

\newpage
\begin{center}
\large
기호 설명
\end{center}
\normalsize
\begin{tabular}{p{.2\textwidth}p{.7\textwidth}}
$M$	& original mass matrix\\
$K$	& original stiffness matrix\\[30pt]
\multicolumn{2}{l}{Subscripts}\\
$b$ & interface boundary\\
$d$ & dominant\\[30pt]
\multicolumn{2}{l}{Abbreviation}\\
$CMS$ & Component Mode Synthesis\\
\end{tabular}
%필요한 경우 기호설명을 작성한다. 기호 설명에는 필요한 경우 첨자 설명 및 약어 설명을 포함한다.
% 본문을 한글로 작성한 경우에도 기호 설명은 영어로 작성 가능하다.
\newpage 

\ % 본문 앞에는 빈페이지를 둔다. 

%%%
\chapter{Introduction}\label{chap:intro}
The following formatting information is intended to illustrate several acceptable ways of preparing a thesis or dissertation for your convenience.

The first level heading is styled using chapter.
Chapter 1 is styled with\\ \verb|\chapter{Introduction}|.
You can put \verb|\label{chap:intro}| to refer to this chapter.

The first paragraph of every chapter, section or subsection is, by default, set to be nonindented.

%%
\section{Second Level Heading}\label{sec:section}
The second level subheading is styled using section.
Section 1.1 is styled with \verb|\section{Second Level Heading}|.
You can put \verb|\label{sec:section}| to refer to this section.

%
\subsection{Third Level Heading}\label{subs:subsection}
The above third level subheading is styled using subsection.
Subsection 1.1.1 is styled with \verb|\subsection{Third Level Heading}|.
You can put \\\verb|\label{subs:subsection}| to refer to this subsection.

It will appear in the Table of Contents, automatically.

%
\section{Referencing headings}\label{sec:referencing}
Suppose that you want to refer to the first section.
The first section (of the first chapter) was labeled with \verb|\label{sec:section}|.
You can call the section by typing \verb|\ref{sec:section}| : Subsection \ref{sec:section}

%%%
\chapter{Organizing and Formatting}\label{chap:organizing}

%%
\section{Paper Size and Margins} \label{sec:papersize}
The paper size of the thesis/dissertation shall be B5.
For the first three preliminary pages (including the cover page, title page and signature page) before the abstract, all margins (top, bottom, left and right) shall be at least 3 cm.
From the abstract on, the top and bottom margins shall be at least 3cm and the left and right margins shall be at least 2 cm (Table \ref{tab:Organizing and formatting}).

\begin{table}[h]\centering
\begin{tabular}{cccc}
\hline
\textbf{Order}&\textbf{Note}&\textbf{Margin}&\textbf{Pagination}\\\hline
Cover page&&\multirow{4}{2.5cm}{\centering top, bottom, left \& right at least 3 cm}&\multirow{4}{2.5cm}{\centering None}\\\cline{1-2}
Blank page&&\\\cline{1-2}
Title page&&\\\cline{1-2}
Signature page&&\\\hline
Abstract&both English \& Korean&\multirow{13}{2.5cm}{\centering top \& bottom at least 3cm\\[\baselineskip] left \& right at least 2 cm}\\\hline
Dedication page&optional&&\multirow{8}{2.5cm}{i, ii, iii, iv, \(\cdots\)}\\\cline{1-2}
Preface&if necessary\\\cline{1-2}
Acknowledgements&optional\\\cline{1-2}
Table of contents&\\\cline{1-2}
List of tables&\multirow{2}{4cm}{\centering if there are tables or figures in the main body}&\\\cline{1-1}
List of figures&&\\\cline{1-2}
Nomenclature&optiona\\\cline{1-2}\cline{4-4}
Blank page&&&None\\\cline{1-2}\cline{4-4}
Main body&&&\multirow{4}{2.5cm}{1, 2, 3, 4, \(\cdots\)}\\\cline{1-2}
Reference&\\\cline{1-2}
Appendices&optional&\\\cline{1-2}
index&optional&\\\hline
\end{tabular}
\caption{Organizing and formatting thesis/dissertation}
\label{tab:Organizing and formatting}
\end{table}

%%
\section{Fonts and Size}\label{sec:font}

The default font size is set to 11pt.
In \LaTeX you can use commands like \verb|\normalsize|, \verb|\large|, \verb|\Large|, \verb|\LARGE|, \verb|\huge|, and so on, to specify the size of the font.
We relate the above commands to 11pt, 14pt, 16pt, 18pt and 21pt, respectly, of the MS word templete.
Thus, there are slight differences of font size in MS word templete and in \LaTeX templete.
The below (Table \ref{tab:font size}) is the comparison table for the font size.
\footnote{\url{https://tug.org/texinfohtml/latex2e.html#Font-sizes}}
\begin{table}[h]\centering
\begin{tabular}{>{\centering\arraybackslash}p{6cm}cc}
\hline
&Size Requirements&\LaTeX Style\\\hline
Thesis title			&21&\verb|\huge|\\\hline
The school name (Graduate School, Korea University)
					&18&\verb|\LARGE|\\\hline
All other parts are 16 points (department, name, advisor, master's thesis, \(\cdots\), submitted, \(\cdots\) completed, etc.)	
					&16&\verb|\Large|\\\hline
Year, month and day	&14&\verb|\large|\\\hline
Main Text			&10--12&\verb|\normalsize|\\\hline
Heading				&None&\\\hline
Figure caption			&None&\\\hline
Table caption			&None&\\\hline
\end{tabular}
\caption{Requirement for font size and the style used in this manuscript}\label{tab:font size}
\end{table}

%\begin{tabular}{ccc}
%MS word templete&\LaTeX commands&\LaTeX templete\\\hline
%11pt&\verb|\normalsize|&10.95pt\\
%14pt&\verb|\large|&12pt\\
%16pt&\verb|\Large|&14.4pt\\
%18pt&\verb|\LARGE|&17.28\\
%21pt&\verb|\huge|&20.74\\
%\end{tabular}
%\caption{Font sizes}\label{tab:font size}

Here is how we put tables and footnote in \LaTeX.
To typeset a table, use the environment \texttt{tabular} and specify the columns.
The above table has three center-aligned columns ;
\begin{verbatim}
\begin{tabular}{ccc} ... \end{tabular}
\end{verbatim}
You can also use advanced version of \texttt{tabular}, which are \texttt{taubularx}, \texttt{tabulary}, \texttt{tabu}, to manipulate the typeset of tables.

It is desirable to put the \texttt{tabular} environment inside the \texttt{table} environment.
You can add caption by \verb|\caption{...}|.
The labeling \verb|\label{...} | for future reference should be followed just after the caption.
All the tables in the \texttt{table} environment will be included in the `List of Tables`.

%%
\section{Figures and Equations}\label{sec:figures_and_equations}

To include the figure file in the document, you can use \texttt{includegraphics} command, which require \texttt{graphicx} package.
\begin{verbatim}
\includegraphics[width=.2\textwidth]{kumark.png}
\end{verbatim}
You can specify the width or the height of the figure inside the square brackets and the file name (with or without the extension) inside the braces.

It is desirable to put the \texttt{includegraphics} command inside the \texttt{figure} environment.
Again, the labeling need to be followed just after the caption.
All the tables in the \texttt{table} environment will be included in the `List of Tables'.

\begin{figure}[h]
\begin{center}
\includegraphics[width=.2\textwidth]{kumark.png}
\end{center}
\caption{Korea University Global Symbol}
\label{fig:kumark}
\end{figure}

You can type an equation with inline math mode like \(E=mc^2\). %or $E=mc^2$.
Or you can type
\[E=mc^2\]
%$$E=mc^2.$$
%\begin{equation*}
%E=mc^2
%\end{equation*}
to express the equation in display math mode.
The above equation is an unnumbered.
To number the equation automatically, you can use \texttt{equation} environment;
\begin{equation}
E=mc^2
\end{equation}
The number or the tag of the above equation reads `the first eqeuation of the section \ref{sec:figures_and_equations}'.
If you add one more equation, you can get the second eqaution of the section \ref{sec:figures_and_equations}.
\begin{equation}
e^{i\theta}=\cos\theta+i\sin\theta.
\end{equation}
You can also specify tha tagging explicitly by
\[E=mc^2\tag{$*$}\]

To express a list of equations, you can use the \texttt{gather} environment, which just enumerate equations vertically.
For example, suppose that you want to express a system of linear equations \(x+y+z=3\), \(x-y+2z=1\), \(x+3z=2\).
\begin{gather}
x+y+z=3\\
x-y+2z=1\\
x+3z=2,
\end{gather}
If you want to unnumber the equations, use \texttt{gather*} environment;
\begin{gather*}
x+y+z=3\\
x-y+2z=1\\
x+3z=2,
\end{gather*}
Note that the above system is not well aligned.
To align the equations horizontally, with respect to the equality sign, you can use \texttt{align} (or \texttt{align*}) environment
\begin{align*}
x+y+z&=3\\
x-y+2z&=1\\
x+3z&=2
\end{align*}
\texttt{align} environment (instead of \texttt{align*} environment) tags every equation of the system
\begin{align}
x+y+z&=3\\
x-y+2z&=1\\
x+3z&=2
\end{align}
If you want one tagging for the system, you can use the \texttt{aligned} environment and the \texttt{equation} environment, simultaneously
\begin{equation}\label{eq:system}
\begin{aligned}
x+y+z&=3\\
x-y+2z&=1\\
x+3z&=2
\end{aligned}
\end{equation}
Finally, you can label and refer an equation, by \verb|\label{...}| and \verb|\eqref{...}|.
For example, you can say that `The root of \eqref{eq:system} is \(x=2\), \(y=1\), \(z=0\)'.

The environments \texttt{align}, \texttt{gather} and others, are the environments provided by the \texttt{amsmath} package.
For more information to typeset the equation neatly, refer to \url{http://www.ams.org/arc/tex/amsmath/amsldoc.pdf}.

%%%
\chapter{Discussion}\label{chap:discussion}
Discussion starts here.

%%%
\chapter{Conclusion}\label{chap:conclusion}
Conclusion starts here. 


\newpage
\renewcommand\bibname{Reference(or Bibliography)}
\addcontentsline{toc}{chapter}{Bibliography}
\begin{thebibliography}{AA}
\bibitem {ACC} C. Adams, M. Chu, T. Crawford, S. Jensen, K. Siegel and L. Zhang,
    {\em Stick index of knots and links in the cubic lattice},
    J. Knot Theory Ramif. \textbf{21} (2012) 1250041.



\end{thebibliography}
\begin{center}
\large
참고문헌
\end{center}
\normalsize
%본문 뒤에는 참고문헌(References) 또는 서지(Bibliography)를 작성한다. 
%
%참고문헌(References)은 본문에서 인용되거나 참고한 자료를 작성한 목록을 말한다. 서지(Bibliography)는 엄밀한 의미에서 참고문헌(References) 뿐만 아니라 음반, 면담, 영화, TV프로그램, 그림 등 비문자 자료를 모두 포괄한다. 
%
%참고문헌은 서지관리 프로그램(Endnote, Mendeley 등)을 사용하여 학문분야 특성에 맞게 저자 이름순 혹은 인용순 등으로 일관된 양식으로 작성한다.
%
%서지관리 프로그램 링크
%\begin{itemize}
%\item\url{https://library.korea.ac.kr/research/writing-guide/endnote/}
%\item\url{https://library.korea.ac.kr/research/writing-guide/mendeley/}
%\end{itemize}
\newpage
\begin{center}
\large
부록
\end{center}
\normalsize
\paragraph{A. 부록 제목}
%필요한 경우 부록(appendices or supplementary materials)을 작성한다.

%부록의 각 장은 영문 알파벳을 사용하여 구분하는 것이 일반적이다.


\newpage
\begin{center}
\large
색인

\end{center}
\normalsize
필요한 경우 색인(index)을 작성한다.
\end{document}